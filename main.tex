\documentclass[a4paper,11pt,oneside,openany]{jsbook}
\usepackage{graphicx,enumerate}
\usepackage{algorithm}
\usepackage{algpseudocode}
\usepackage{float}
\usepackage{amssymb}
\usepackage{mathabx}
\usepackage{longtable}
\usepackage{supertabular}
\usepackage{subfigure}
\usepackage{lscape}

% declaration of the new block
\algblock{ParallelForByBlocks}{EndParallelForByBlocks}
% customising the new block
\algnewcommand\algorithmicparallelforbyblocks{\textbf{parallelForByBlocks}}
\algnewcommand\algorithmicpardo{\textbf{do}}
\algnewcommand\algorithmicendparallelforbyblocks{\textbf{end\ parallelForByBlocks}}
\algrenewtext{ParallelForByBlocks}[1]{\algorithmicparallelforbyblocks\ #1\ \algorithmicpardo}
\algrenewtext{EndParallelForByBlocks}{\algorithmicendparallelforbyblocks}

\pagestyle{plain}
\setlength{\textwidth}{\fullwidth}
\setlength{\evensidemargin}{\oddsidemargin}
\def\vector#1{\mbox{\boldmath $#1$}}
\begin{document}
\thispagestyle{empty}
%------------------------------標題紙作成エリア----------------------------%
2017年度 修士学位論文%1
\bigskip%2
\LARGE%3
\begin{center}
修士学位論文
\end{center}
\bigskip\bigskip\bigskip\bigskip\bigskip\bigskip\bigskip %7
\begin{center} %8
PDBを用いたGPU上の並列IDA*探索
\end{center}
\large %11
\begin{center}
Parallel IDA* search using PDB for GPUs 
\end{center}
\bigskip\bigskip\bigskip\bigskip\bigskip\bigskip\bigskip\bigskip\bigskip\bigskip
\bigskip\bigskip\bigskip\bigskip\bigskip\bigskip\bigskip\bigskip\bigskip
\Large %17
\begin{center}
広域システム科学系 広域科学専攻
\end{center}
\Large %17
\begin{center}
指導教員: 福永 アレックス
\end{center}
\LARGE %21
\begin{center}
山村 武史
\end{center}
\normalsize
%---------------------------------目次エリア-------------------------------%
\thispagestyle{empty}
\tableofcontents
%---------------------------------本文エリア-------------------------------%

\chapter{序論}
\section{研究の概要}
本研究では、ヒューリスティック探索の一つであるIDA*探索\cite{Kor85}をGPUを利用した並列化手法についての研究に取り組んだ。
IDA*探索についてGPUを利用した高速な並列アルゴリズムのひとつにBPIDA*探索\cite{HA17}がある。この手法では、IDA*探索を一つのブロックに対してノードを一つ割り当て、このノードを根とするf値制限深さ優先探索を行うことで、warp divergenceを隠蔽し、高速な並列化を行った。これによって直列実行時と比べて5倍ほどのスピードアップに成功している。

堀江らの研究\cite{HA17}におけるBPIDA*探索では、ヒューリスティック関数としてマンハッタン距離を使用し、これを共有メモリにおいて使用していた。共有メモリは、GPUカーネル側に置くことのできるオンチップメモリであり、カーネル側から高速にアクセスすることができる一方、そのメモリ容量は、64Kbyteと比較的小さい。

ヒューリスティック探索の探索性能は、ヒューリスティック関数の精度に依存する。N-puzzle問題における、ヒューリスティック関数のうちマンハッタン距離より精度の高いものとして、パターンデータベースがあげられる。パターンデータベースは例えば$N=24$の時のN-puzzle問題では、250Mbyteほどの容量を必要とし、マンハッタン距離に比べ、かなり多くのメモリを使用する必要がある。共有メモリの大きさが64Kbyteであることを考慮すると、パターンデータベースをBPIDA*探索で使用するには、当然より大きなメモリ領域を使用しなければならない。
本研究では、パターンデータベースを共有メモリよりも、容量が大きく、低速なグローバルメモリに置くことで、直列実行時のIDA*探索と、BPIDA*探索のスピードアップの比較を行った。これにより、共有メモリではなくグローバルメモリを使用することによる遅延がBPIDA*探索の探索時間にどのような変化をしめすのか検証した。

またこの検証結果を踏まえ、BPIDA*探索における共有メモリに置くことで一つのブロックで共有していたスタックをグローバルメモリへ移動させたところ、当初の予想と反し、5.41倍高速化を行うことができた。また同時にグローバルメモリに


% \section{本研究の目的}
% 本研究の目的はぶらぶらぶら
% \begin{enumerate}
% \item ぶらぶら
% \vspace{3mm}
% \newline
% ぶらぶらぶらぶらぶら
% \newline

% \item ぶらぶらぶら
% \vspace{3mm}
% \newline
% ぶらぶらぶらぶら
% \end{enumerate}


\section{本論文の構成}
本論文は以下の通りに構成される。

2章で、本研究の背景としてGPUおよび、GPU上でのプログラミングに必要なプラットフォームであるCUDAについての説明を行った後、本研究で取り扱うグラフ探索問題について問題の定義と、それを取り扱う古典的なアルゴリズムについて説明する。

3章では、本研究と関連した先行研究について紹介をする。

4章では、本研究においてベースラインとなる直列環境および並列環境における、先行研究についての再現実験について紹介する。

5章では、グローバルメモリにPattern Databaseをおいた直列環境および並列環境におけるIDA*探索の実験について紹介する。

6章では、5章での知見をふまえ、共有メモリが大きな領域を確保した際にBPIDA*探索がどのような影響をうけるか、その検証結果を紹介する。

7章では、本研究における知見をまとめる。

\chapter{導入}
% \chapter{GPUとグラフ探索アルゴリズム}
\section{GPUとCUDA}
GPU(graphics processing unit)は2Dや3Dといったコンピューターグラフィックスの画像処理を行う為に従来使用され、それらの性能を向上することを追求してきた。
GPUをグラフィックスのレンダリングのみならず、より汎用的な計算にも利用することを目的とした技術がGPGPU(General-purpose computing on graphics processing units)である。
GPGPUが使われる分野として、物理シミュレーション、機械学習、音声処理、金融工学などのアプリケーションがあげられる。現在、機械学習への注目が高まるとともに、その需要が高まっている。


CUDA(Compute Unified Device Architecture)とはNVIDIA社が提供するGPGPUのための統合開発環境である。GPUの並列コンピュートエンジンを利用して多くの複雑なコンピューティング問題をより効率的に解決する。CUDAではC、C++、Fortranなどの言語で開発が可能である。
本節では、GPUを利用したCUDAプログラミングにおける実行単位、およびメモリモデルについて説明する。

\subsection{実行単位}
% ブロック スレッド グリッドの説明
%ホストとデバイスの説明

\subsection{メモリモデル}
プログラミングを行う上でメモリは一般的に、プラグラマー側が明示的に制御可能なプログラマブルなメモリと、データの配置を制御することが不可能なノンプログラマブルなメモリの二つに分類される。CUDAのメモリモデルでは、レジスタ、共有メモリ、ローカルメモリ、コンスタントメモリ、テクスチャメモリ、グローバルメモリの6種類が制御可能なプログラマブルメモリであり、GPUのキャッシュであるL1キャッシュ、L2キャッシュ、コンスタントキャッシュ、テクスチャキャッシュの4種類がノンプログラマブルなメモリである。

またCUDAにおけるメモリはさらにGPUの内部に存在するオンチップメモリと、オフチップメモリに分かれる。
オンチップメモリには高速なアクセスが可能であが、容量が小さい。一方、オフチップメモリではアクセスは低速だが、その容量は大きいという特徴がある。
ここでは、本論文において重要な意味をもつグローバルメモリ、共有メモリおよびL1キャッシュについて詳しく説明する。

グローバルメモリは、オフチップメモリの一つで、GPUにおいて最も容量が大きく、最も遅延が大きいメモリである。グローバルメモリでは、すべてのスレッド、ホストから読み書きが可能であり、SMの外部に用意されている。

共有メモリは、オンチップメモリの一つで、一つのブロック内の全てのスレッドが共有する記憶領域であり、グローバルメモリと比べて、GPU側からの高速な読み書きが可能となっている。一方その容量は比較的小さく、一つのブロックあたり最大で64Kbyteほどしか確保できない。

L1キャッシュはノンプログラマブルメモリの一つで、ローカルメモリとグローバルメモリにアクセス時のデータをキャッシュするために使用される。そのメモリ容量は共有メモリと64Kbyteのメモリ領域を共有する。

レジスタはGPUにおいて最も高速なオンチップメモリであり、カーネルで宣言される自動変数は通常レジスタに登録される。レジスタの数がハードウェア制限を超える場合、超えた分のレジスタはローカルメモリに退避される。このような現象をレジスタピルと呼び、パフォーマンスに悪影響を及ぼす可能性がある。


% \section{CUDA}
% CUDA(Compute Unified Device Architecture)とはNVIDIA社が提供するGPUコンピューティングのための統合開発環境である。。GPUの並列コンピュートエンジンを利用して多くの複雑なコンピューティング問題をより効率的に解決する。CUDAではC、C++、Fortranなどの言語で開発が可能である。本節ではCUDAプログラミングにおける

% \subsection{GPUのメモリ構造}
% GPUの大きな特徴の一つとしてCPUとは、異なるメモリ構造があげられる。GPUのデバイス側で使われるメモリは大きくGPU側のオンチップメモリと、CPU側のオフチップメモリの2つに別れる。オンチップメモリの方が、カーネル側からは高速なアクセスが可能である一方、そのサイズは小さくなる。大容量であるメモリのほとんどは、CPU側のオンチップメモリに存在し、容量と引き換えに、高遅延となっている。
% 変数のスコープやライフタイムについても、それぞれのメモリで異なり、これらの特徴をまとめると、以下の図のようになっている。GPUを使用したプログラミングでは、これらのメモリの特徴を踏まえた上で、適したメモリを選択することが、高速化の肝となる。
% 特にグローバルメモリと共有メモリについては本論文において重要な役割を果たすため、紹介する。
% グローバルメモリはCPU上において最も容量が大きい一方で遅延が最も大きいメモリである。すべてのスレッドとホストの両方から読み書きが可能である。
% それに対して、共有メモリはオンチップメモリの一つであり、同じブロック同士でのみスコープを共有する。


% \subsection{CUDA}
% CUDA(Compute Unified Device Architecture)とはNVIDIA社が提供する汎用的な並列コンピューティングプラットフォームとAPIを含む、プログラミングモデルであり、NVIDIAによって開発された。GPUの並列コンピュートエンジンを利用して多くの複雑なコンピューティング問題をより効率的に解決する。CUDAではC、C++、Fortranなどの言語で開発が可能であり、
% CUDAではCPUとそのメモリを扱うホストと、GPUとそのメモリを扱うデバイスの二つにわかれている、ヘテロジニアスアーキテクチャである。

% \subsection{実行単位}
% CUDAプログラミングの特徴の一つとして
% \begin{enumerate}
% \item ホスト
% \vspace{3mm}
% \newline
% ぶらぶらぶらぶらぶら
% \newline

% \item デバイス
% \vspace{3mm}
% \newline
% ぶらぶらぶらぶら
% \end{enumerate}

\section{GPUとCUDA}

\section{グラフ探索アルゴリズム}
\section{グラフ探索問題}
グラフとはノードの集合$V$と、そのノード同士を結ぶコストを持ったエッジの集合$E$からなる。またこのエッジに方向を定義している場合、そのようなグラフを有向グラフとよび、一方エッジに方向の定義していないグラフを無向グラフと呼ぶ。
グラフ探索問題の一つである最短コスト探索問題は、このグラフの中から、入力として、初期ノード$v_0$とゴールノード$v_g$があったときに、$v_0$から$v_g$に至る為の最小の経路を出力する問題である。

\section{ヒューリスティック探索}

\section{A*探索}
A*探索\cite{HNR68}はヒューリスティック関数と呼ばれる、ある状態からゴール状態までの見積もりを道しるべに使用したグラフ探索の手法の一つである。
初期状態から現在までの移動コストと、現在から終了状態までのコストの見積もり値が小さなノードから順に探索を行うことで、ゴール状態までの経路を探索することができる。
終了状態までの見積もりであるヒューリスティック関数による値が、実際にゴール状態に至るまでの最短移動コスト以下であるとき、このようなヒューリスティック関数をadmissibleであると呼ぶ。
admissibleなヒューリスティック関数を使用した場合、A*探索では必ず最短経路を発見することが保証される。
また本研究で使用するManhattan DistanceやPattern Data Baseはこのadmissibleなヒューリスティックである。
A*探索の探索効率は、このヒューリスティック関数に依存し、ヒューリスティック関数が常に正しい値を返すことができる場合、分岐をすることなく最短経路を発見することができる。このような場合をパーフェクトヒューリスティックと呼ぶ。
またヒューリスティック値が常に0である時、A*探索はダイクストラ探索と同じアルゴリズムとなる。
A*探索は適切なヒューリスティック関数を用いることでダイクストラ法よりも効率良く、最短経路問題を解くことが可能である一方、メモリの使用量の大きさがネックとなる。
ヒューリスティック関数
closedlistによる重複検知
admissible
ダイクストラ
ダイクストラのヒューリスティック値0
探索効率はhの正確性に左右

\newpage
\begin{algorithm}
\caption{A*}
\label{alg:pbnf}
\begin{algorithmic}[1]
\State $openList$を見積もり値$f(v)$を優先度としたプライオリティーキューとして定義
\State 初期ノード$v_0$を$openList$に追加。
\State $closedList$をハッシュテーブルとして定義
\While {$openList$が空ではない}
    \State $v_{cur}$ $\leftarrow$ $openList$の中で $f(v_{cur})$が最も小さいノード
    \State $openList$から$v_{cur}$を削除
    \If {$v_{cur} \in 終了状態$}
        \State $v_0$から$v_{cur}$までの遷移と、移動コスト$g(v_{cur})$を解として出力して終了
    \EndIf
    \State $closedList$に$v_{cur}$を追加
    \For{each $v_{next}$ $\in$ $children(v_{cur})$ do}
        \State $f_{new}(v_{next}) \leftarrow g(v_{cur}) + cost(v_{cur}, v_{next}) + h(v_{next})$
        \If {$v_{next}\not\in openList \land v_{next}\not\in closedList$}
            \State $f(v_{next}) \leftarrow f_{new}(v_{next})$ 
            \State $openList$に$v_{next}$を追加
            \State $v_{next}$の親を$v_{cur}$として記録
        \EndIf
        \If {$v_{next}\in openList \land f_{new}(v_{next}) < f(v_{next})$}
            \State $f(v_{next}) \leftarrow f_{new}(v_{next})$
            \State $openList$から$v_{next}$を削除し、再び$v_{next}$を追加することで、優先度を更新
            \State $v_{next}$の親を$v_{cur}$として記録
        \EndIf
        \If {$v_{next}\in closedList \land f_{new}(v_{next}) < f(v_{next})$}
            \State $f(v_{next}) \leftarrow f_{new}(v_{next})$
            \State $closedList$から$openList$に$v_{next}$を移動
            \State $v_{next}$の親を$v_{cur}$として記録
        \EndIf
    \EndFor
\EndWhile
\State 初期状態から終了状態までのパスは存在しないと出力して終了
\end{algorithmic}
\end{algorithm}
\newpage

\section{IDA*探索}
IDA*探索\cite{Kor85}はA*探索アルゴリズムのメモリ不足による探索失敗を回避するために提案された手法である。空間計算量がエッジの数と比例するA*探索に対し、IDA*探索では探索の深さに対して線形を保つことができる。IDA*ではA*と同様にヒューリスティック関数によってyotte状態の評価を行い、展開の順序を決定するアルゴリズムである。

\newpage
\begin{algorithm}
\caption{IDA*}
\label{alg:pbnf}
\begin{algorithmic}[1]
\State $limit_f \leftarrow 0$
\While {true}
    \State $openList$を空のスタックとして初期化
    \State 初期ノード$v_0$を$openList$に追加
    \State $f_{next} \leftarrow \infty$
    \State $f_{next}$は次回の$limit_f$更新時に利用される 
    \While {$openList$が空ではない}
        \State $v_{cur}$ $\leftarrow$ $openList$から先頭ノードを取り出す
        \State $openList$から$v_{cur}$を削除
        \If {$v_{cur} \in 終了状態$}
            \State $v_0$から$v_{cur}$までの遷移と、移動コスト$g(v_{cur})$を解として出力して終了
        \EndIf
        \For{each $v_{next}$ $\in$ $children(v_{cur})$ do}
            \State $f_{new} \leftarrow g(v_{cur}) + cost(v_{cur}, v_{next}) + h(v_{next})$
            \If {$f_{new}(v_{next}) \leqq limit_f$}
                \State $openList$に$v_{next}$を追加
            \Else
                \State $f_{next} \leftarrow min(f_{next}, f_{new})$
            \EndIf
        \EndFor
    \EndWhile
    \State $update(limit_f)$
\EndWhile
\end{algorithmic}
\end{algorithm}
\newpage



% \chapter{GPUを使用した並列IDA*探索の}
\chapter{先行研究}
\section{PSimple}

\section{Block Parallel IDA*}
GPUを使用したIDA*探索の並列化の一つにBlock-Parallel IDA*(BPIDA*)\cite{Horie17}がある。この手法では、同じBlockが1つのスタックを共有することでblock並列化を行っている。一つのスタックを共有するため、スタックからノードを取り出す、および追加する操作はそれぞれ、並列に行わなければならない。特にスタックにノードを追加する操作、AtomicPutでは、スタックにロックをかけ、原子性を保ちながら行う。

同一ブロック内の全てのスレッドが共有する一つのスタックが$sharedOpenList$である。このスタックには二つの並列な操作、$parallelPop$と$atomicPut$を行う。$parallelPop$では、sharedOpenListから1ブロック内のスレッド数を行為の数で割った数だけのノードを一度に取り出す。$atomicPut$では$v_{next}$を$sharedOpenList$に同時に挿入する。この操作を行う際には、挿入をする前に$sharedOpenList$にロックをかけ、挿入後そのロックを外すことで原子性をたもつ。
BPDFS関数は標準的なf値制限深さ優先探索とにている。


\newpage
\begin{algorithm}
\caption{Block Parallel IDA*}
\label{alg:pbnf}
\begin{algorithmic}[1]
\Function{BPDFS}{$root, goals, limit_f$}
    \State $sharedOpenList$を空のスタックとして初期化し共有メモリにおく
    \State 初期ノード$root$を$sharedOpenList$に追加
    \State $f_{next} \leftarrow \infty$
    \State $f_{next}$は次回の$limit_f$更新時に利用される 
    \While {$sharedOpenList$が空ではない}
        \State $v_{cur}$ $\leftarrow$ ${ParallelPop}(sharedOpenList)$
        \If {$v_{cur} \in goals$}
            \State $root$から$v_{cur}$までの遷移と、移動コスト$g(v_{cur})$を解として出力して終了
        \EndIf
        \State $v_{next} \gets $$v_{cur}$を($threadId$を行為数で割った余り)番目の行為で遷移した時の状態
        \State $f_{new} \leftarrow g(v_{cur}) + cost(v_{cur}, v_{next}) + h(v_{next})$
        \If {$f_{new}(v_{next}) \leqq limit_f$}
            \State ${AtomicPut}(sharedOpenList, v_{next})$ \Comment{openListに$v_{next}$を原子性を保ちつつ追加}
        \Else
            \State $f_{next} \leftarrow min(f_{next}, f_{new})$
        \EndIf
    \EndWhile
    \State $update(limit_f)$

    \State \Return $a$
\EndFunction
\Function{BPIDA*}{$start, goals$}
    \State $rootSet \gets {CreateRootSet}(start, goals)$
    \State $limit_f \leftarrow {DecideFirstLimit}(rootSet)$
    \While {最短距離が発見されるまで}
        \ParallelForByBlocks{$each root \in rootSet do$}
            \State $limit_f, stat \gets {BPDFS}(root, goals, limit_f)$
        \EndParallelForByBlocks
        \State $UpdateRootSet(rootSet, stat)$
    \EndWhile
\EndFunction

\end{algorithmic}
\end{algorithm}
\newpage





\begin{table}[]
\centering
\caption{Table 1: Total Runtimes for 50 24-Puzzle Instances}
\label{my-label}
\begin{tabular}{|l|c|}
\hline
configuration & \multicolumn{1}{l|}{total runtime(seconds)} \\ \hline
\multicolumn{2}{|l|}{CPU-based sequential algorithms} \\ \hline
IDA* (manhattan distance) & 743.019 \\
IDA* (PDB) & 1.4446 \\ \hline
\multicolumn{2}{|l|}{GPU-based parallel algorithms} \\ \hline
PSimple (manhattan distance) & 2106.27 \\
PSimple (PDB) & 4.91605 \\
BPIDA* (manhattan distance) & 674.521 \\
BPIDA* (PDB) & 0.994512 \\ \hline
\end{tabular}
\end{table}

\begin{table}[]
\centering
\caption{Table 1: Total Runtimes for 50 24-Puzzle Instances}
\label{my-label}
\begin{tabular}{|l|c|}
\hline
configuration & \multicolumn{1}{l|}{total runtime(seconds)} \\ \hline
\multicolumn{2}{|l|}{CPU-based sequential algorithms} \\ \hline
IDA* (manhattan distance) & 743.019 \\
IDA* (PDB) & 1.4446 \\ \hline
\multicolumn{2}{|l|}{GPU-based parallel algorithms} \\ \hline
PSimple (manhattan distance) & 2106.27 \\
PSimple (PDB) & 4.91605 \\
BPIDA* (manhattan distance) & 674.521 \\
BPIDA*GLOBAL (manhattan distance) & 124.586 \\
BPIDA* (PDB) & 0.994512 \\
BPIDA*GLOBAL (PDB) & 0.401973 \\ \hline
\end{tabular}
\end{table}

\begin{table}[]
\centering
\caption{Table 1: Total Runtimes for 50 24-Puzzle Instances}
\label{my-label}
\begin{tabular}{|l|c|}
\hline
configuration & \multicolumn{1}{l|}{total runtime(seconds)} \\ \hline
\multicolumn{2}{|l|}{CPU-based sequential algorithms} \\ \hline
IDA* (manhattan distance) & 743.019 \\
IDA* (PDB) & 1.4446 \\ \hline
\multicolumn{2}{|l|}{GPU-based parallel algorithms} \\ \hline
PSimple (manhattan distance) & 2106.27 \\
PSimple (PDB) & 4.91605 \\
BPIDA* (manhattan distance) & 674.521 \\
BPIDA*GLOBAL (manhattan distance) & 124.586 \\
BPIDA* (PDB) & 0.994512 \\
BPIDA*GLOBAL (PDB) & 0.401973 \\ \hline
\end{tabular}
\end{table}

\chapter{終わりに}
\section{まとめ}



%-----------------------------参考文献記述エリア---------------------------%
\begin{thebibliography}{10}
  \bibitem{HA17} Satoru Horie and Alex Fukunaga. Block-Parallel IDA* for GPUs. In SOCS, 2017 
  \bibitem{Burns et al. 2012} Burns, E. A.; Hatem, M.; Leighton,  M. J.; and Ruml, W. 2012. Implementing fast heuristic search code. In SOCS.
  \bibitem{BHLR12}Ethan Andrew Burns, Matthew Hatem, Michael J Leighton, and Wheeler Ruml. Implementing fast heuristic search code. In SOCS, 2012.
  \bibitem{Kor85}Korf, R. E. 1985. Depth-first iterativedeepening: An optimal admissible tree search. Artificial intelligence 27(1):97-109.
  \bibitem{KF02}Korf, R. E., and Felner, A. 2002. Disjoint pattern database heuristics. Artificial intelligence 134(1):9-22.
  \bibitem{HNR68}Peter E Hart, Nils J Nilsson, and Bertram Raphael. A formal basis for the heuristic determination of minimum cost paths. IEEE transactions on Systems Science and Cybernetics, Vol. 4, No. 2, pp. 100-107, 1968. 

\end{thebibliography}
%---------------------------------必須エリア-------------------------------%
\end{document}


%----------------------------ファイルはここまで----------------------------%
